\documentclass{article}
\usepackage[english]{babel}
\usepackage{a4wide}
\usepackage{color}
\usepackage[colorlinks,linkcolor=blue]{hyperref}
\bibliographystyle{unsrt}

\newcommand{\myhref}[2]{\textcolor{blue}{\href{#1}{\textcolor{blue}{#2}}}}

\title{Jean-Jacques Rousseau's influence on Immanuel Kant\\ \textit{\small Essay for the course on the Modern and the Postmodern @ Wesleyan University}}
\author{Vitaly Repin}
\date{August 2013}

\begin{document}
\maketitle

\textit{``Enlightenment is the project to make the world more of a home for human beings --- through the use of reason''} --- this is one of the contemporary definitions of
Enlightenment~\cite{Roth}.
But how did the  people of that epoch understand themselves the era they lived in? Let's focus on the two very famous philosophers of the period --- Immanuel Kant (1724 -- 1804) and
Jean-Jacques Rousseau (1712 -- 1778). What was common and what was different in their understanding of the period? How one influenced another?


Immanuel Kant had defined Enlightenment as \textit{''man's emergence from his self-incurred immaturity. Immaturity is the inability to use one's own understanding without the guidance
of another''.} He emphasizes that \textit{''this immaturity is self-incurred if its cause is not lack of understanding, but lack of resolution and courage to use it without the guidance of another''}
and suggests the famous motto of enlightenment: \textbf{\textit{``Sapere aude! Have courage to use your own understanding!''}}~\cite{Kant}.

This motto is very crucial to understand the Kant's view. Were men able to use their own understanding without the guidance of another prior to the Enlightenment? Absolutely!
Plato and Julius Caesar in Antiquity, Occam and Christopher Columbus in Medieval period and many, many others. However, these people were exceptions. And Enlightenment appeals
to much wider audience --- not to political leaders, geniuses and other outstanding persons anymore. Kant believes that the regular people should have courage to use their
own understanding (if they have it). This habit shall not be an exception, it shall be common in order for the world to become better.

Jean-Jacques Rousseau was one the favorite Kant's authors. It sounds surprising as Rousseau did not share the belief into power of reasoning and was not enthusiast of a progress
(rather opposite instead).
He stated his motives to write the famous ``Discourse on the Origin and Basis of Inequality Among Men'' in the autobiographic book ``Confessions''~\cite{Confessions}:

\begin{quote}
I confounded the pitiful lies
of men; I dared to unveil their nature; to follow the progress of time, and the things by which it has been disfigured; and comparing the man of art with the natural man, to show them, in
\textbf{their pretended improvement, the real source of all their misery.}
\end{quote}

In this discourse he attacks thinking in general: \textit{``I venture to declare that a state of reflection is a state contrary to nature, and that a thinking man is a depraved animal.''}~\cite{discourse2}. He was attacking reasoning by reasoning (utilizing all the reasoning tools of modernity) and fully understood this. The closing words of the discourse on inequality sound a little bit ironic~\cite{discourse2}:

\begin{quote}
I have endeavoured to trace
the origin and progress of inequality, and the institution and abuse of political societies, as far as these are capable of being deduced from the nature of man
\textbf{merely by the light of reason, and independently of those sacred dogmas} which give the sanction of divine right to sovereign authority.
\end{quote}

Voltaire's (1694 -- 1778) reaction to Rousseau was something you can expect from one of the most famous reasoning' advocates of all times --- he was very sarcastic about Rousseau's works.
Kant's reaction was totally different --- no denial and no sarcasm but incorporation into his own philosophical system. Instances of direct influence include Kant's idea of
the categorical imperative. Rousseau's influence can also be seen in Kant's moral psychology, in Kant's own thinking about conjectural history, and in his writings on international
justice~\cite{stanford}.

These famous Kant's words show how important was Rousseau's influence on Kant's ethical principles~\cite{propagator}:
\begin{quote}
I am by natural inclination a researcher ... and I thought that this alone could constitute the honor of man ... \textbf{Rousseau set me upright.} And I would consider myself more useless than
the ordinary worker if everything I did did not contribute to securing the rights of man.
\end{quote}

However, Kant was not just blind ``fun'' of Rousseau. He criticized Rousseau's ``synthetic'' method (begins with man in the natural state which never existed)
and describes his own method as ``analytic'' as it examines man in the civilized condition~\cite{clewis}.

Kant was trying to steer a middle course for philosophy~\cite{Roth}. He tried to find a place for the moral principles in a brave new world of rational thinkers. And Rousseau's attention
to the moral problems was the call which Kant was not able to ignore. Rousseau was a good diagnostician while Kant tried to find the cure for the illness. He had introduced the concepts
of practical and pure reasons, categorical imperative, thing-in-self (\textsl{das Ding an sich}) and phenomenon (\textsl{Erscheinung}) which made a place for morals in the rational world of modernity.

\bibliography{kantrousseau}

\end{document}
